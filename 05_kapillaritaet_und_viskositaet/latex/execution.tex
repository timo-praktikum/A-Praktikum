\section{Durchführung}
\begin{figure}[!htbp]
\centering
\resizebox{0.8\textwidth}{!}{\input{versuchsskizze.pdf_tex}}
\caption{Versuchsskizze: Kreisel mit Ausgleichsgewicht $m_a$ und Zusatzgewicht $m_z$}
\label{img:3}
\end{figure}\ \\
Abbildung \ref{img:3} zeigt schematisch den Versuchsaufbau. Der Kreisel besteht aus einem Rad, welches an einer Achse gelagert ist. Mit dem Ausgleichsgewicht $m_a$ kann die Achse in die horizontale Gleichgewichtslage gebracht werden (Verschiebung des Schwerpunkts zum Punkt der Aufhängung). Die Achse kann durch eine Halterung fest eingespannt werden. Am Ende der Achse befindet sich eine Kerbe zum Einhängen der Zusatzgewichte $m_z$.
\subsection{Physikalisches Pendel}
Der Kreisel wird an der Achse eingespannt. Am Rad des Kreisels wird ein Zusatzgewicht befestigt, um dieses als physikalisches Pendel auszubilden. Die Schwingungsdauer des Pendels wird über mehrere Perioden wiederholt gemessen. Anschließend wird das Gewicht an der diametral gegenüberliegenden Stelle befestigt und die Messung wiederholt.
\subsection{Präzession}
Die Einspannung wird entfernt und der Kreisel in die horizontale Gleichgewichtslage gebracht, indem das Ausgleichsgewicht $m_a$ entsprechend verschoben wird. Am Rad wird ein Papierstreifen befestigt, um die Rotationsperiode $T_R$ mit einer Lichtschranke messen zu können. Mit einer Aufzugsschnur wird das Rad in schnelle Rotation versetzt und die Rotationsperiode mit der Lichtschranke gemessen. Anschließend wird ein Zusatzgewicht $\SI{10}{\g} \leq m_z \leq \SI{60}{\g}$ angehängt und der Kreisel vorsichtig in die Präzessionsbewegung eingeführt. Die halbe Präzessionsperiode $T_P/2$ wird mit einer Stoppuhr gemessen. Nach dem halben Umlauf wird das Zusatzgewicht entfernt und erneut die Rotationsperiode $T_R$ gemessen. Das Zusatzgewicht wird danach wieder angehängt und wieder die Präzessionsperiode gemessen. Dies wird insgesamt vier Mal wiederholt.\\
Für zwei weitere Gewichte $m_z$ wird der beschriebene Vorgang wiederholt.
\subsection{Nutation}
Wie im Präzessionsversuch wird der Kreisel in schnelle Rotation versetzt und zunächst die Rotationsperiode $T_R$ gemessen. Anschließend wird auf die Achse ein Stoß ausgeübt, um den Kreisel in Nutation zu versetzen. Mit der Stoppuhr wird die Nutationsperiode $T_N$ über mehrere Perioden gemessen. Nachdem die Nutation abgeklungen ist, wird erneut die Rotationsperiode mit der Lichtschranke gemessen und ein weiterer Stoß auf die Achse ausgeübt. Dies wird insgesamt drei Mal wiederholt.\\
\\
Zusätzlich werden alle Parameter gemessen, die für die Berechnung des Trägheitsmomentes des Kreisels benötigt werden (Längen, Abstände, Massen).