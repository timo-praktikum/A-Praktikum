\section{Einleitung}
\subsection{Motivation}
Anhand des im Folgenden beschriebenen Versuchs mit dem Pohlschen Resonator werden mechanische Schwingungen,
mit und ohne Anregung und insbesondere auch Resonanzen untersucht. Die dabei zu lösenden Schwingungsgleichungen sind 
sehr wesentlich in der Physik, da sie in gleicher oder ähnlicher Form in sehr vielen Gebieten auftreten.
Auch das in diesem Experiment vertiefte Verständnis der erwähnten Resonanzerscheinungen sowie die Überlagerung von 
Schwingungen ist auf viele Gebiete der Physik, von der Astrophysik bis hin zur Atomphysik, übertragbar.   

\subsection{Überblick}
Im folgenden Versuch wird die Schwingung eines \glqq Pohlsches Rades\grqq{} \footnote[1]{nach Robert Wichard Pohl(1884-1976, 
deutscher Physiker)} computergestützt aufgezeichnet und analysiert. Im ersten Versuchsteil geschieht die Auslenkung per Hand,
im zweiten durch einen automatischen Exzenter.
