\section{Theorie}
\subsection{Die gedämpfte erzwungene Schwingung}
In der homogenen Differentialgleichung des gedämpften harmonischen Oszillators gibt es drei wesentliche Größen: Das der
(Winkel-)Beschleunigung entgegenwirkende Trägheitsmoment $\Theta$ (hier das Schwungrad), die zur (Winkel-)Geschwindigkeit proportionale Reibung
$\rho$ und das direkt zur Position (hier der Winkel $\phi$) proportionale Rückstellmoment $D^*$.
Hinzu kommt die Anregung. Diese wird periodisch beschrieben durch $M\cos(\omega t).$ 
Zusammen ergeben diese Faktoren die Bewegungsgleichung des Pohlschen Rades:
\begin{align}
\label{eq:1}
\Theta\ddot{\varphi}+\rho\dot{\varphi}+D^*=M\cos(\omega t)
\end{align}
Um diese Gleichung auf die Normalform einer Bewegungsgleichung für eine gedämpfte Schwingung mit Anregung zu bringen, teilen
wir durch $\Theta$ und erhalten für $\rho/\Theta=:2\beta$, $D^*/\Theta=:\omega_0^2$ und $M/\Theta=:N$
die inhomogene lineare Differentialgleichung 2. Ordnung
\begin{align}
\label{eq:2}
\ddot{\varphi}+2\beta\dot{\varphi}+\omega_0^2\varphi=N\cos(\omega t).
\end{align}


\subsection{Die homogene Differentialgleichung}
\label{h.DGL}
Betrachten wir zunächst die Gleichung des frei schwingenden Rades, also ohne Anregung, so erhalten wir die homogene Differentialgleichung
\begin{align}
\label{eq:3}
\ddot{\varphi}+2\beta\dot{\varphi}+\omega_0^2\varphi=0
\end{align}
welche mit dem Exponentialansatz $\varphi(t)=e^{\lambda t}$ gelöst werden kann. Hier sind die drei Fälle $\beta>\omega_0$, $\beta=\omega_0$ und $\beta<\omega_0$ zu unterscheiden. 
Wir werden uns hier mit dem sogenannten Schwingfall $\beta<\omega_0$ befassen. 
Offensichtlich bekommt hier die Wurzel ein negatives Argument, was zu folgender Lösung für
Gleichung (\ref{eq:3}) führt:
\begin{align}
\label{eq:4}
\varphi(t)=e^{-\beta t}\cdot\left(Ae^{i\sqrt{\omega_0^2-\beta^2}t} + Be^{-i\sqrt{\omega_0^2-\beta^2}t}\right)
\end{align}
Nun ist noch die Anfangsphase $\phi$, sowie die Reellen Zahlen A und B, welche 
in der Ausgangsamplitude $\varphi_0$ zusammengefasst werden zu bestimmen. Setzen wir $A:=\frac{\varphi_0}{2}\cdot e^{i\phi}$ und 
$B:=\frac{\varphi_0}{2}\cdot e^{-i\phi}=\bar{A}$ so erhalten wir \begin{align}
\label{eq:5}
\varphi(t) &= e^{-\beta t}\cdot\left(\frac{\varphi_0}{2}\cdot e^{i\phi}\cdot e^{i\sqrt{\omega_0^2-\beta^2}t} + 
\frac{\varphi_0}{2}\cdot e^{-i\phi}\cdot e^{-i\sqrt{\omega_0^2-\beta^2}t}\right)\nonumber \\
&= e^{-\beta t}\cdot\frac{\varphi_0}{2}\cdot\left(e^{i(\phi+\sqrt{\omega_0^2-\beta^2}t)}+e^{-i(\phi+\sqrt{\omega_0^2-\beta^2}t)}\right).
\end{align}
Mit $\sqrt{\omega_0^2-\beta^2}=:\omega_e$, der Eigenfrequenz des Rades bei der entsprechenden Schwingung, folgt nach der 
\glqq Eulerschen Identität \grqq{} die Schwingungsgleichung für das Pohlsche Rad ohne Antrieb.
\begin{align}
\label{eq:6}
\varphi(t)=\varphi_0\cdot e^{\beta t}\cdot\cos(\omega_e + \phi)
\end{align}

\subsection{Interpretation und logarithmisches Dekrement}
\label{log.Dek}
Der Schwingungsverlauf des Rades ist also cosinus-periodisch und wird um den von der Zeit abhängigen Faktor $e^{-\beta t}$ geschwächt. 
Diese  Dämpfung kann über das logarithmische Dekrement $\Lambda$ beschrieben werden. Dabei gilt:
\begin{align}
\label{eq:7}
\Lambda := \rm {ln}\left[{\frac{\varphi(t)}{\varphi(t+T)}}\right]={\rm {ln}}[e^{\beta T}]=\beta T
\end{align}
Hier beschreibt T die Periodendauer. Das logarithmische Dekrement hängt also lediglich von der Periodendauer, nicht 
von der Zeit ab.

\subsection{Die inhomogene Differentialgleichung}
\label{inh.DGL}
Um die Gleichung der erzwungenen Schwingung zu finden, müssen wir nun die inhomogene Gleichung (\ref{eq:2}) lösen. 
Die Lösung einer inhomogenen Gleichung setzt sich aus der Lösung der zugehörigen homogenen 
Differentialgleichung (\ref{eq:6}) und einer partikulären Lösung zusammen. 
Die partikuläre Lösung erhalten wir durch Einsetzen des Ansatzes $\varphi=\varphi_0\cdot \cos(\omega t-\phi)+c$ in 
(\ref{eq:2}):
\begin{align}
\label{eq:8}
(\omega_0^2-\omega^2)\cdot\varphi_0\cdot\cos(\omega t-\phi)-2\beta\cdot\omega\cdot\sin(\omega t-\phi)+c=N\cos(\omega t). 
\end{align}
Nun müssen $\varphi_0$ und $\phi$ so bestimmt werden, dass die Gleichung für alle Zeiten t erfüllt ist.
Mithilfe der Additionstheoreme gelangt man von dieser Gleichung zu folgenden zwei Bedingungen:
\begin{align}
\label{eq:9}
\varphi_0\cdot\left((\omega_0^2-\omega^2)\cdot\cos(\phi)+2\beta\cdot\omega\cdot\sin(\phi)\right) &= N \\
\label{eq:10}
(\omega_0^2-\omega^2\cdot\sin(\phi) &= 2\beta\cdot\omega\cdot\cos(\phi)
\end{align}
Aus (\ref{eq:10}) folgt direkt die Phasenverschiebung:
\begin{align}
\label{eq:11}
\phi=\arctan\left(\frac{2\beta\omega}{\omega_0^2-\omega^2}\right)
\end{align}
Aus (\ref{eq:9}) lässt sich nach kurzem Umformen und Einsetzen der Phasenverschiebung $\varphi_0$ bestimmen:
\begin{align}
\label{eq:12}
\varphi_0=\frac{N}{(\omega_0^2-\omega^2)^+4\beta^2\cdot\omega^2}
\end{align}
Einsetzen dieses Ergebnisses führt zu folgender partikuläreN Lösung der Schwingungsgleichung (\ref{eq:2}) für die gedämpfte erzwungene Schwingung 
des Pohlschen Rades:
\begin{align}
\label{eq:13}
\varphi(t)=\frac{N}{(\omega_0^2-\omega^2)^+4\beta^2\cdot\omega^2}\cdot\cos\left(\omega t-\arctan\left(\frac{2\beta\omega}{\omega_0^2-\omega^2}\right)\right)
\end{align}
Da die Lösung der homogenen Differentialgleichung für lange Messzeiten gegen Null geht (die Amplitude fällt mit der Exponentialfunktion $e^{-\beta t}$ 
ab), kann für die erzwungene Schwingung bei längeren Schwingzeiten die homogene Gleichung vernachlässigt werden.
Damit beschreibt die partikuläre Lösung der inhomogenen Differentialgleichung die Schwingung für große Zeiten t zunehmend genau. Die Zeit, nach der die homogene Gleichung mit Null angenähert werden kann bezeichnet man als Einschwingvorgang.
\newpage
\subsection{Die Amplitudengleichung}
\label{ampl}
Die Amplitudenstärke
\begin{align}
\label{eq:14}
A(\omega,\beta)=\frac{N}{(\omega_0^2-\omega^2)^2+4\beta^2\cdot\omega^2}
\end{align}
ist gegeben als Vorfaktor der Schwingungsgleichung (\ref{eq:13}) in Abhängigkeit von der Erregerfrequenz $\omega$ und der Dämpfung $\beta$.
Da wir das Resonanzverhalten der Schwingung bei einer bestimmten Dämpfung betrachten wollen, gehen wir jeweils von einem festen $\beta$ aus 
und bestimmen die Amplitude in Abhängigkeit von der Anregung.\\
Dabei ist das Betrachten des Maximums besonders relevant. Dafür berechnen wir die Nullstellen der Ableitung der 
Amplitudengleichung. Die relevante Nullstelle dieser Ableitung ist
\begin{align}
\label{eq:16}
\omega=\sqrt{\omega_0^2-2\beta^2}=:\omega_r.
\end{align}
$\omega_r$ ist die Resonanzfrequenz des Systems. Entspricht die Anregungsfrequenz der Resonanzfrequenz, des angeregten 
Systems, so wird die Schwingsamplitude maximal. In diesem Fall, wird das schwingende System immer weiter angeregt, 
bis es schließlich (abhängig von der Dämpfung) im Extremfall zu einer sogenannten Resonanzkatastrophe
kommt. Als Resonanzkatastrophe wird der Zustand bezeichnet, bei welchem die Amplitude des schwingenden Systems gegen 
unendlich geht. 

\subsection{Phasenverschiebung}
In Kapitel \ref{inh.DGL} wurde die Gleichung (\ref{eq:11}) der Phasenverschiebung
\begin{align}
\phi=\arctan\left(\frac{2\beta\omega}{\omega_0^2-\omega^2}\right) \nonumber
\end{align}
hergeleitet.
Anhand dieser Gleichung ist zu sehen, wie sich unterschiedliche Dämpfungen, beziehungsweise Anregungen auf die 
Phasenverschiebung $\phi$ auswirken.