\section{Der Versuch}
\subsection{Versuchsaufbau}
Die Versuchsapparatur besteht im Wesentlichen aus drei Elementen: Einem Rechner mit Bildschirm, dem Pohlschen Rad mit 
Wirbelstrombremse und einem Schrittmotor mit Exzenter als externem Anreger für das Rad.

\subsection{Durchführung}
Für vier Stellungen der Wirbelstrombremse ($\SI{0}{mm}$, $\SI{4}{mm}$, $\SI{6}{mm}$ und $\SI{8}{mm}$) wird das Verhalten der freien Schwingung aufgezeichnet. Anschließend werden für die drei Stellungen der Wirbelstrombremse $\SI{4}{mm}$, $\SI{6}{mm}$ und $\SI{8}{mm}$
jeweils Messungen für Anregungsfrequenzen im Bereich von 100-600 mHz durchgeführt. Insbesondere werden hier zusätzliche 
Messungen in der Umgebung der Resonanzfrequenz aufgezeichnet.\\
Bei der Messung der Schwingung mit Anregung ist es wichtig, die Einschwingzeit (vgl. Kap. \ref{inh.DGL}) zu berücksichtigen, um eine Verfälschung der 
Messdaten zu verhindern. Besondere Vorsicht ist bei Messungen nahe der Resonanzfrequenz (vgl. Kap. \ref{ampl}) geboten. Außerdem ist es für die
Auswertung des Versuches wichtig, hinreichend viele Messungen insbesondere um den Resonanzbereich zu tätigen, da sonst der 
Verlauf in diesem Bereich nur sehr ungenau dargestellt werden kann.

