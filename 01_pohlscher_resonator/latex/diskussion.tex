\section{Diskussion}
Insgesamt war es spannend einen solchen Schwingungsprozess nochmal nachzuvollziehen und auszuwerten. Etwas erfolgreichere
Messergebnisse hätten allerdings die Motivation noch gesteigert.

\subsection{Fehlerdiskussion}
Da eventuelle Ungenauigkeiten bei der Aufnahme der Daten nicht angegeben waren, haben wir uns an der Auflösung 
der aufgezeichneten Daten orientiert, hierbei sind wir von einer \glqq fehlerfreien \grqq{} Zeitbestimmung 
ausgegangen. 
Der wesentlichste vermeidbare Fehler war gegebenenfalls zu kurzes Warten bei der Einschwingzeit, was
die Messergebnisse verfälscht haben kann. \\
Deutlich hat sicherlich der Zustand der Versuchsapparatur zu Fehlern beigetragen, was insbesondere an signifikant 
unterschiedlichen Messergebnissen an den drei Apparaturen zu sehen ist.
Auch die Tatsache dass auch bei kleinster Dämpfung die Amplitude der erzwungenen Schwingung nicht in die Nähe einer 
Resonanzkatastrophe kam spricht für Ungenauigkeiten durch Verschleiß der Apparaturen. \\
Bei der Auswertung der Messwerte hätte man die Fehler beim Logarithmischen Dekrement (sowie den davon abhängigen Werten) minimieren können, indem man weniger Maxima einbezogen (der Fehler nimmt mit kleinerer Amplitude zu) oder ein gewichtetes Mittel verwendet hätte. \\
Eine weitere Auffälligkeit ist die Verschiebung der Resonanzkurven (vgl. Abb. \ref{img:2}). Vor allem die Kurve der starken Dämpfung ist deutlich nach rechts verschoben. Die Ursache dafür könnte entweder im Verschleiß des Messaufbaus oder in der Ungenauigkeit bei der Bestimmung der Eigenfrequenz liegen. Möglicherweise ist auch die Angabe der Exzenterfrequenzen fehlerbehaftet.