\section{Fehlerrechnung}
\subsection{Aufnahme der Werte}
Da die Güte der verwendeten Messgeräte nicht bekannt ist, muss der Messfehler geschätzt werden. Der Auslenkungswinkel des Pohlschen Rades wird mit einer Auflösung von $\SI{0,25}{\degree}$ aufgezeichnet, daher wird ein Fehler von $\sigma_\varphi=\SI{0,125}{\degree}$ angenommen.

\subsection{Logarithmisches Dekrement}
Der Fehler bei der Berechnung des Logarithmischen Dekrements lässt sich durch eine Gaußsche Fehlerfortpflanzung ermitteln:

\begin{align}
\sigma_\Lambda=\sqrt{\sigma_{\varphi_k}^2\cdot\left( \frac{1}{\varphi_k}\right)^2+\sigma_{\varphi_{k+1}}^2\cdot\left( \frac{1}{\varphi_{k+1}}\right) ^2}
\end{align}
Da die Werte gemittelt werden, wird jeweils der größte Fehler angenommen.
\ \\
Mit einer weiteren Fehlerfortpflanzung folgt der Fehler der Dämpfungskonstante $\beta$

\begin{align}
\sigma_\beta=\sqrt{\sigma_\Lambda^2}
\end{align}

\subsection{Eigenfrequenz}
Der Fehlerfortpflanzung der ungedämpften Eigenfrequenz lautet
\begin{align}
\sigma_{\omega_0} = \sqrt{\sigma_{\omega_e}^{2}\left(\frac{\omega_e}{\sqrt{\omega_e^{2}+\beta^{2}}}\right)^{2}+\sigma_\beta^{2}\left(\frac{\beta}{\sqrt{\omega_e^{2}+\beta^{2}}}\right)^{2}}
\end{align}

\subsection{Resonanzfrequenz}
Die Gaußsche Fehlerfortpflanzung ergibt für die Resonanzfrequenz einen Fehler von
\begin{align}
\sigma_{\omega_r}=\sqrt{\sigma_{\omega_0}\frac{\omega_e}{\sqrt{\omega_e^{2}-2\beta^{2}}}+\sigma_\beta\frac{2\beta}{\sqrt{\omega_e^{2}-2\beta^{2}}}}
\end{align}

\subsection{Diskrete Fouriertransformation}
Die DFT besitzt nur eine begrenzte Auflösung, die von der Anzahl der Messwerte $N$ und der Abtastzeit $t_a$ abhängt. Die Auflösung x berechnet sich zu
\begin{align}
x = \frac{1}{N t_a}
\end{align}
Zu den Frequenzen der DFT addiert sich somit ein Fehler in der Größe der halben Auflösung. In diesem Versuch ist $t_a = \SI{15,625}{ms}$.