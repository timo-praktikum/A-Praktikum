\section{Durchführung}
\subsection{Versuchsteil A}
Im ersten Versuchsteil werden Größen der Drehschwingung gemessen, um daraus das Trägheitsmoment verschiedener Körper zu berechnen.
\subsubsection{Messung der Körpereigenschaften}
Zunächst werden Eigenschaften der Körper gemessen, die für die Auswertung benötigt werden. Tabelle \ref{tab:1} gibt Aufschluss, was für welchen Körper gemessen wird.
\begin{table}[!htpb]
\begin{tabular}{|l|l|}
\hline 
\textbf{Körper} & \textbf{zu messende Größen} \\ 
\hline 
Kugel & Radius, Masse \\ 
\hline 
Zylinder & Radius, Masse \\ 
\hline 
Scheibe & Radius, Masse \\ 
\hline 
Hohlzylinder & innerer Radius, äußerer Radius, Masse \\ 
\hline 
Hantel & Abstand der Hantelkörper, Masse \\ 
\hline 
Würfel & Kantenlänge, Masse \\ 
\hline 
Stab & Länge, Abstand der Drehachse vom Schwerpunkt, Masse \\ 
\hline 
\end{tabular}
\caption{Gemessene Größen der verschiedenen Körper}
\label{tab:1} 
\end{table}
\subsubsection{Vermessung der Drehschwingungen}
Für diesen Versuchsteil wird eine Spiralfeder mit einer Haltevorrichtung für die Versuchskörper verwendet. Zunächst soll die Winkelrichtgröße der Vorrichtung bestimmt werden. Dazu wird der Halter so eingespannt, dass die Achse horizontal liegt und der Winkelausschlag in Abhängigkeit des angreifenden Drehmomentes gemessen. Die Drehmomente (jeweils nach links und rechts) werden durch angehängte Gewichte erzeugt. Anschließend wird der Halter vertikal eingespannt und die Schwingungsdauer für jeden Körper (bei einigen Körpern für verschiedene Achsen) gemessen.\\
Außerdem wird ein Tischchen-förmiger Körper (Platte mit zwei diagonal gegenüberliegenden Beinen) untersucht. Das Trägheitsmoment dieses Körpers kann durch Drehung an der Symmetrieachse verändert werden. Die Schwingungsdauer wird für in $\SI{15}{\degree}$-Schritten verdrehte Rotationsachsen gemessen.
\subsection{Versuchsteil B}
Der zweite Versuchsteil beschäftigt sich mit einem Speichenrad, welches sowohl frei rotierend, als auch als physikalisches Pendel ausgebildet werden kann.
\subsubsection{Messung der Winkelbeschleunigung}
Mit einem Faden werden verschiedene Massen ($\SI{0,1}{\kg}$, $\SI{0,2}{\kg}$, $\SI{0,5}{\kg}$, $\SI{1}{\kg}$) so am Rad befestigt, dass diese beim Herabfallen ein Drehmoment verursachen. Mit einem Markengeber (Frequenz $\SI{10}{\hertz}$) werden Zeitmarken der Bewegung aufgezeichnet, um daraus die Winkelbeschleunigung zu bestimmen. Um daraus das Trägheitsmoment berechnen zu können werden der Radius der Felge und der Radius des Zusatzrades, an dem der Faden befestigt ist, gemessen.
\subsubsection{Physikalisches Pendel}
An einer Speiche des Rades wird ein zusätzliches Gewicht befestigt, sodass die Anordnung ein physikalisches Pendel darstellt. Für kleine Amplituden wird die Schwingungsdauer gemessen. Anschließend wird das Gewicht an der gegenüberliegenden Speiche befestigt und die Messung wiederholt. Für die Auswertung müssen außerdem die Masse des Zusatzgewichtes und der Abstand des Gewichtes von der Drehachse bestimmt werden.