\section{Theorie}
\subsection{Drehimpuls und Drehmoment}
Betrachten wir einen starren Körper als Sytem von Massenpunkten mit Ortsvektoren $\vec{r}_i$ und den Einzelimpulsen $\vec{p}_i$. Dann ist der \textbf{Drehimpuls} $\vec{L}$ definiert als
\begin{align}
\vec{L}:=\sum\limits_{i}\vec{r}_i\times\vec{p}_i.
\end{align}
Die zeitliche Ableitung des Drehimpulses bezeichnet man als \textbf{Drehmoment}:
\begin{align}
\dot{\vec{L}}=\vec{M}
\end{align}
\subsection{Trägheitsmoment}
Ein Massenelement $\Delta{m_i}$ eines mit der Winkelgeschwindigkeit $\omega$ rotierenden Körpers habe den Abstand $r_{i\perp}=\left|\vec{r}_i\right|$ von der Drehachse. Seine Geschwindigkeit $\vec{v}_i\perp\vec{r}_i$ ist
\begin{align}
\vec{v}_i=\vec{\omega}\times\vec{r}_i.
\end{align}
Die kinetische Energie eines solchen Elementes ist
\begin{align}
E_{kin}\left(\Delta{m_i}\right)=\frac{1}{2}\Delta{m_i}v^2_i=\frac{1}{2}\Delta{m_i}r^2_{i\perp}\omega^2
\end{align}
und die Rotationsenergie des Körpers ergibt sich durch Aufsummieren aller kinetischen Energien:
\begin{align}
E_{rot}&=\lim\limits_{\substack{n\to\infty\\\Delta{m_i}\to0}}\left(\frac{1}{2}\sum\limits_{i=1}^{N}\Delta{m_i}r^2_{i\perp}\omega^2\right)\nonumber\\
&=\frac{1}{2}\omega^2\int \, r^2_{\perp}\, \mathrm{d}m
\end{align}
Das darin auftauchende Integral definiert man als \textbf{Trägheitsmoment}:
\begin{align}
\Theta:=\int\limits_{V} \, r^2_{\perp}\, \mathrm{d}m=\int\limits_{V} \, r^2_{\perp}\rho\, \mathrm{d}V
\end{align}
Dieses ist von der Rotationsachse abhängig, wobei es für jeden Körper drei \textbf{Hauptträgheitsmomente} $\Theta_A, \Theta_B, \Theta_C$ entlang der Hauptachsen gibt, aus denen sich für jede beliebige Achse das Trägheitsmoment berechnen lässt.\\
Für die Drehimpulse entlang der Hauptachsen gilt
\begin{align}
L_A=\Theta_A\omega_A; \, L_B=\Theta_B\omega_B; \, L_C=\Theta_C\omega_C.
\end{align}
\subsection{Satz von Steiner}
Kennt man das Trägheitsmoment einer durch den Schwerpunkt verlaufenden Achse A, so lässt sich daraus leicht das Trägheitsmoment einer parallelen Achse B berechnen. Dabei gilt der \textbf{Steinersche Satz}:
\begin{align}
I_B=I_A+a^2M\\
a:=& \text{Abstand A-B}\nonumber\\
M:=& \text{Gesamtmasse}\nonumber
\end{align}
\subsection{Bestimmung der Trägheitsmomente aus Drehschwingungen}
Steht eine der Hauptträgheitsachsen (hier exemplarisch C) senkrecht auf der Rotationsachse, so gilt
\begin{align}
\label{eq:1}
\Theta\omega^2=\vec{L}\cdot\vec{\omega}=\Theta_A\omega^2_A+\Theta_B\omega^2_B
\end{align}
Setzt man $\omega_A=\omega\cos{\alpha}, \, \omega_B=\omega\cos{\beta}$ in Gleichung \ref{eq:1} ein, erhält man die Ellipsengleichung
\begin{align}
\frac{\xi^2}{a^2}+\frac{\eta}{b^2}=1
\end{align}
mit $\Theta_A=\frac{1}{a^2}, \, \Theta_B=\frac{1}{b^2}, \, \xi=\frac{\cos{\alpha}}{\sqrt{\Theta}}, \, \eta=\frac{\cos{\beta}}{\sqrt{\Theta}}$. Kennt man also die Haupt- und Nebenachse der Ellipse, kann man daraus die Hauptträgheitsmomente berechnen.
\subsection{Bestimmung des Trägheitsmomentes aus dem Drehmoment}
Zur Bestimmung des Trägheitsmomentes kann ein Körper mit der zu untersuchenden Achse an einer Feder befestigt und die Schwingungsdauer der zugehörigen Drehschwingung gemessen werden. Bei Auslenkung um den Winkel $\varphi$ übt eine Spiralfeder das rücktreibende Drehmoment
\begin{align}
\label{eq:2}
M=-D\varphi
\end{align}
aus. $D$ bezeichnet man als \textbf{Winkelrichtgröße} (oder Direktionsmoment). Gleichzeitig gilt
\begin{align}
\label{eq:3}
M=\Theta\ddot{\varphi}.
\end{align}
Gleichsetzen von \ref{eq:2} und \ref{eq:3} ergibt eine Differentialgleichung, deren Lösung eine Beziehung zwischen Trägheitsmoment, Winkelrichtgröße und Schwingungsdauer liefert:
\begin{align}
\Theta=D\cdot\left(\frac{T}{2\pi}\right)^2
\end{align}
\subsection{Bestimmung des Trägheitsmomentes aus der Winkelbeschleunigung}
An einem Rad sei ein Schwungrad befestigt, welches durch eine herabfallende Masse $m$ beschleunigt wird. Auf das Schwungrad wird dabei die Kraft $F=mg-ma'$ ausgeübt. Dadurch wirkt das Drehmoment
\begin{align}
T=mgr-ma'r
\end{align}
und man erhält für das Trägheitsmoment
\begin{align}
\Theta=\frac{mgrR-ma'rR}{a}.
\end{align}
Mit der Beziehung $\frac{a}{R}=\frac{a'}{r}$ ergibt sich
\begin{align}
\Theta=\frac{mgrR}{a}-mr^2
\end{align}
\subsection{Trägheitsmoment des Physikalischen Pendels}
Wird das Pendel um kleine Winkel $\varphi$ ausgelenkt, wirkt näherungsweise die Rückstellkraft
\begin{align}
F_r=-m_kg\varphi
\end{align}
Einsetzen des Drehmomentes $M=-F_{r}r_s$ ergibt die Differentialgleichung
\begin{align}
\Theta\ddot{\varphi}=-r_sm_kg\varphi
\end{align}
Die Lösung liefert
\begin{align}
\Theta=\frac{T^2r_sm_kg}{4\pi^2}
\end{align}
Betrachten wir ein schwerpunktgelagertes Rad der Masse $m_r=m_k-m$ mit der Punktmasse $m$ im Abstand $z$ vom Mittelpunkt. Dann erhält man für das Trägheitsmoment des Rades
\begin{align}
\Theta_r=\frac{T^2zmg}{4\pi^2}-mz^2.
\end{align}